\documentclass{article}
\usepackage[utf8]{inputenc}
\usepackage[russian]{babel}
\usepackage[T1]{fontenc}

\title{Weighted shuffle}
\date{2019-10-17}
\author{Borogin Gregory}

\newtheorem{theorem}{Теорема}
\newtheorem{problem}{Задача}

\begin{document}
    \pagenumbering{gobble}
    \maketitle
    \newpage
    \pagenumbering{arabic}

    \section{Абстрактное описание задачи}

    Необходимо реализовать алгоритм weighted shuffle, который получает $n$~элементов на вход,
    соответствующие веса $w_i, 1\leq{}i\leq{}n,$ и возвращает полученный список элементов в некотором порядке.
    Обозначим вероятность того, что $k$-й элемент окажется на $j$-м месте $O_{k,j}$. Тогда должно быть верно
    $$O_{k,1}=\frac{w_k}{\sum_{i=1}^{n} w_i}, 1\leq{}k\leq{}n,$$
    то есть вероятность оказаться на первом месте для элемента с весом $w_k$ равна отношению этого веса
    к сумме весов всех элементов.

    Вероятности попадания на другие места явно не заданы, но хотелось бы, чтобы эти вероятности были
    выбраны каким-то логичным образом.

    \section{Предложенное решение}

    Сопоставим каждому элементу $L_i$ некоторую случайную величину $X_i$.
    Все случайные величины $X_i, 1\leq{}i\leq{}n,$ независимы в совокупности.
    Вид распределения каждой такой случайной величины будет одинаковый,
    но параметры будут некоторым образом зависеть от весa $w_i$ соответствующего элемента $L_i$.

    Каждый раз, когда необходимо выполнить алгоритм, необходимо взять конкретные реализации $x_i$
    объявленных случайных величин $X_i$, отсортировать их в порядке возрастания,
    а затем упорядочить элементы $L_i$ так, чтобы их порядок совпадал с порядком реализаций $x_i$.

    Утверждается, что полученная сортировка является искомой в случае, если случайные величины имеют вид
    $$X_i\in{}Exp[w_i],$$
    то есть имеют экспоненциальное распределение с параметром равным весу соответствующего элемента.

    \section{Доказательство}



\end{document}